%% Generated by Sphinx.
\def\sphinxdocclass{report}
\documentclass[letterpaper,10pt,spanish]{sphinxmanual}
\ifdefined\pdfpxdimen
   \let\sphinxpxdimen\pdfpxdimen\else\newdimen\sphinxpxdimen
\fi \sphinxpxdimen=.75bp\relax
\ifdefined\pdfimageresolution
    \pdfimageresolution= \numexpr \dimexpr1in\relax/\sphinxpxdimen\relax
\fi
%% let collapsable pdf bookmarks panel have high depth per default
\PassOptionsToPackage{bookmarksdepth=5}{hyperref}

\PassOptionsToPackage{warn}{textcomp}
\usepackage[utf8]{inputenc}
\ifdefined\DeclareUnicodeCharacter
% support both utf8 and utf8x syntaxes
  \ifdefined\DeclareUnicodeCharacterAsOptional
    \def\sphinxDUC#1{\DeclareUnicodeCharacter{"#1}}
  \else
    \let\sphinxDUC\DeclareUnicodeCharacter
  \fi
  \sphinxDUC{00A0}{\nobreakspace}
  \sphinxDUC{2500}{\sphinxunichar{2500}}
  \sphinxDUC{2502}{\sphinxunichar{2502}}
  \sphinxDUC{2514}{\sphinxunichar{2514}}
  \sphinxDUC{251C}{\sphinxunichar{251C}}
  \sphinxDUC{2572}{\textbackslash}
\fi
\usepackage{cmap}
\usepackage[T1]{fontenc}
\usepackage{amsmath,amssymb,amstext}
\usepackage{babel}



\usepackage{tgtermes}
\usepackage{tgheros}
\renewcommand{\ttdefault}{txtt}



\usepackage[Sonny]{fncychap}
\ChNameVar{\Large\normalfont\sffamily}
\ChTitleVar{\Large\normalfont\sffamily}
\usepackage{sphinx}

\fvset{fontsize=auto}
\usepackage{geometry}


% Include hyperref last.
\usepackage{hyperref}
% Fix anchor placement for figures with captions.
\usepackage{hypcap}% it must be loaded after hyperref.
% Set up styles of URL: it should be placed after hyperref.
\urlstyle{same}

\addto\captionsspanish{\renewcommand{\contentsname}{Capítulos:}}

\usepackage{sphinxmessages}
\setcounter{tocdepth}{1}



\title{TFG \sphinxhyphen{} Sistema de recomendaciones para accesos a estadios de fútbol}
\date{20 de mayo de 2021}
\release{1.0.0}
\author{Pelayo Tiesta Cosío}
\newcommand{\sphinxlogo}{\vbox{}}
\renewcommand{\releasename}{Versión}
\makeindex
\begin{document}

\ifdefined\shorthandoff
  \ifnum\catcode`\=\string=\active\shorthandoff{=}\fi
  \ifnum\catcode`\"=\active\shorthandoff{"}\fi
\fi

\pagestyle{empty}
\sphinxmaketitle
\pagestyle{plain}
\sphinxtableofcontents
\pagestyle{normal}
\phantomsection\label{\detokenize{index::doc}}



\chapter{Introducción}
\label{\detokenize{Introduccion:introduccion}}\label{\detokenize{Introduccion::doc}}

\section{Descripción}
\label{\detokenize{Introduccion:descripcion}}
\sphinxAtStartPar
Este Sistema, desarrollado en Python 3, ha sido creado como un código solución complementario al TFG \sphinxhyphen{} Sistemas de recomendaciones para accessos a estadios de fútbol.
Dicho TFG trata de definir y crear un sistema de recomendaciones para los usuarios que desean acceder a un estadio de futbol, de forma que se brinden a estos diferentes soluciones de rutas
en función de las necesidades de cada espectador.

\sphinxAtStartPar
Este sistema ofrece 4 salidas o rutas distintas al usuario:
\begin{itemize}
\item {} 
\sphinxAtStartPar
Rutas más corta desde la puerta de entrada hasta el asiento.

\item {} 
\sphinxAtStartPar
Rutas más rapia desde la puerta de entrada hasta el asiento.

\item {} 
\sphinxAtStartPar
Rutas más corta y que maximice las necesidades del usuario.

\item {} 
\sphinxAtStartPar
Ruta con simulación de control de aglomeraciones.

\end{itemize}


\section{Instalación}
\label{\detokenize{Introduccion:instalacion}}
\sphinxAtStartPar
El código no require de instalación pero si si de una serie de pasos de configuraciones para poder ejecutarlo.


\subsection{1. Instalar Python3:}
\label{\detokenize{Introduccion:instalar-python3}}
\sphinxAtStartPar
Python debe estar instlado en el equipo para poder ejecutar el programa.

\sphinxAtStartPar
Se puede descargar desde la \sphinxhref{https://www.python.org/downloads/}{Página oficial de Python}


\subsection{2. Instalar librerías:}
\label{\detokenize{Introduccion:instalar-librerias}}
\sphinxAtStartPar
Se deben instalar la siguientes librarías para poder ejecutar el programa:

\sphinxAtStartPar
Matplotlib

\begin{sphinxVerbatim}[commandchars=\\\{\}]
\PYGZdl{} pip install matplotlib
\end{sphinxVerbatim}

\sphinxAtStartPar
Networkx
.. code:: bash
\begin{quote}

\sphinxAtStartPar
\$ pip install networkx
\end{quote}

\sphinxAtStartPar
Colorama

\begin{sphinxVerbatim}[commandchars=\\\{\}]
\PYGZdl{} pip install colorama
\end{sphinxVerbatim}


\subsection{3. Distribución de ficheros y carpetas:}
\label{\detokenize{Introduccion:distribucion-de-ficheros-y-carpetas}}
\sphinxAtStartPar
Los ficheros deben estar colocados de la siguiente forma:

\begin{sphinxVerbatim}[commandchars=\\\{\}]
CODIGO
├── Main.py
├── Funciones.py
├── Clases.py
└── Ficheros
        ├── Sectores.csv
        ├── Nodos.json
\end{sphinxVerbatim}


\section{Ejecución}
\label{\detokenize{Introduccion:ejecucion}}
\sphinxAtStartPar
Para ejecutar el progrma se debe lanzar el programa Main.py

\sphinxAtStartPar
En Windows:

\sphinxAtStartPar
Doble click en el fichero Main.py

\sphinxAtStartPar
Desde la CMD

\begin{sphinxVerbatim}[commandchars=\\\{\}]
\PYGZdl{} python Main.py
\end{sphinxVerbatim}

\begin{sphinxVerbatim}[commandchars=\\\{\}]
\PYGZdl{} python3 Main.py
\end{sphinxVerbatim}

\sphinxAtStartPar
En Linux:

\begin{sphinxVerbatim}[commandchars=\\\{\}]
\PYGZdl{} python Main.py
\end{sphinxVerbatim}

\begin{sphinxVerbatim}[commandchars=\\\{\}]
\PYGZdl{} python3 Main.py
\end{sphinxVerbatim}


\section{Uso del programa}
\label{\detokenize{Introduccion:uso-del-programa}}
\sphinxAtStartPar
Al ejecutar el programa Main.py:
\begin{enumerate}
\sphinxsetlistlabels{\arabic}{enumi}{enumii}{}{.}%
\item {} 
\sphinxAtStartPar
Se muestra el grafo del estadio al usuario.

\end{enumerate}

\sphinxAtStartPar
2.El usuario debe introducir los datos de su asiento:
\begin{itemize}
\item {} 
\sphinxAtStartPar
Sector \sphinxhyphen{} Con un formato SXX (Donde XX es el numero que identifica al sector)

\item {} 
\sphinxAtStartPar
Fila

\item {} 
\sphinxAtStartPar
Columna

\item {} 
\sphinxAtStartPar
Puerta de entrada

\end{itemize}

\begin{sphinxadmonition}{note}{Nota:}
\sphinxAtStartPar
Si el sector o la columna esta fuera del rango de filas o columnas del sector, se producirá una excepción.
\end{sphinxadmonition}

\begin{sphinxadmonition}{note}{Nota:}
\sphinxAtStartPar
Actualmente solo se permite la puerta de entrada “P1”.
\end{sphinxadmonition}
\begin{enumerate}
\sphinxsetlistlabels{\arabic}{enumi}{enumii}{}{.}%
\setcounter{enumi}{2}
\item {} 
\sphinxAtStartPar
Se selecciona el tipo de ruta que se desea obtener.

\end{enumerate}

\begin{sphinxadmonition}{note}{Nota:}
\sphinxAtStartPar
Si el tipo es  “Rutas más corta y que maximice las necesidades del usuario”, se abrirá una ventana para solicitar las preferencias del usuario.
\end{sphinxadmonition}


\chapter{Librerías}
\label{\detokenize{Librerias:librerias}}\label{\detokenize{Librerias::doc}}

\section{Librerías utilizadas en el proyecto:}
\label{\detokenize{Librerias:librerias-utilizadas-en-el-proyecto}}

\subsection{Matplotlib:}
\label{\detokenize{Librerias:matplotlib}}
\sphinxAtStartPar
\sphinxhref{https://matplotlib.org/stable/contents.html}{Matplotlib} es una biblioteca completa para crear visualizaciones estáticas, animadas e interactivas en Python.

\sphinxAtStartPar
El uso principal en este proyecto es representar de forma gráfica los nodos que componen la grada del estadio. De esta forma es posible entender de manera más intuitiva las rutas recomendadas,
así como facilitar la depuración tanto del modelo de la grada, como de las soluciones ofrecidas al usuario.


\subsection{Networkx:}
\label{\detokenize{Librerias:networkx}}
\sphinxAtStartPar
\sphinxhref{https://networkx.org/documentation/stable/index.html}{Networkx} es una biblioteca para la creación, manipulacion y estudio de grafos y redes.

\sphinxAtStartPar
El uso principal en este proyecto es la creación de grafos, compuestos por enlaces que simulen los diferentes pasillos del estadio asi como de las uniones de estos. De este modo
se podrá visualizar el grafo de forma gráfica, además de realizar sobre este modificaciones, lecturas y aplicaciones de algoritmos para la obtención de diferentes rutas.


\subsection{tkinder:}
\label{\detokenize{Librerias:tkinder}}
\sphinxAtStartPar
\sphinxhref{https://docs.python.org/3/library/tk.html}{tkinder} es modulo Python que permite crear ý mostrar interfaces grficas al usuario.

\sphinxAtStartPar
El uso principal en este proyecto será la creación de un formulario para leer las preferencias de la ruta del espectador.


\subsection{Numpy:}
\label{\detokenize{Librerias:numpy}}
\sphinxAtStartPar
\sphinxhref{https://numpy.org/doc/}{Numpy} es una biblioteca que facilita y acelerar el trabajo con datos vectoriales y matrices en Python.


\subsection{json:}
\label{\detokenize{Librerias:json}}
\sphinxAtStartPar
\sphinxhref{https://docs.python.org/3/library/threading.html}{threading} es un modulo Python utilizado para condificar y descodificar estructuras de datos en formato .json.


\subsection{threading:}
\label{\detokenize{Librerias:id5}}
\sphinxAtStartPar
\sphinxhref{https://docs.python.org/3/library/threading.html}{threading} es un modulo Python utilizado para crear hilos de paralelismo en los que realizar diferentes tareas simultaneamente.


\chapter{Funciones}
\label{\detokenize{Funciones:funciones}}\label{\detokenize{Funciones::doc}}
\sphinxAtStartPar
Módulo con las funciones usadas en el proyecto.

\phantomsection\label{\detokenize{Funciones:module-Funciones}}\index{módulo@\spxentry{módulo}!Funciones@\spxentry{Funciones}}\index{Funciones@\spxentry{Funciones}!módulo@\spxentry{módulo}}
\sphinxAtStartPar
@author: Pelayo Tiesta
\index{DibujarGrafoGeneral() (en el módulo Funciones)@\spxentry{DibujarGrafoGeneral()}\spxextra{en el módulo Funciones}}

\begin{fulllineitems}
\phantomsection\label{\detokenize{Funciones:Funciones.DibujarGrafoGeneral}}\pysiglinewithargsret{\sphinxcode{\sphinxupquote{Funciones.}}\sphinxbfcode{\sphinxupquote{DibujarGrafoGeneral}}}{\emph{\DUrole{n}{Grafo}}, \emph{\DUrole{n}{mostrarPesos}}, \emph{\DUrole{n}{sectores}}, \emph{\DUrole{n}{peso}\DUrole{o}{=}\DUrole{default_value}{\textquotesingle{}weight\textquotesingle{}}}}{}
\sphinxAtStartPar
Función para dibujar el grafo General
\begin{quote}\begin{description}
\item[{Parámetros}] \leavevmode\begin{itemize}
\item {} 
\sphinxAtStartPar
\sphinxstyleliteralstrong{\sphinxupquote{Grafo}} (\sphinxstyleliteralemphasis{\sphinxupquote{networkx.classes.multidigraph.MultiDiGraph}}) \textendash{} Grafo a dibujar.

\item {} 
\sphinxAtStartPar
\sphinxstyleliteralstrong{\sphinxupquote{mostrarPesos}} (\sphinxstyleliteralemphasis{\sphinxupquote{bool}}) \textendash{} Boolean \sphinxhyphen{} True si se muestran los Pesos

\item {} 
\sphinxAtStartPar
\sphinxstyleliteralstrong{\sphinxupquote{sectores}} (\sphinxstyleliteralemphasis{\sphinxupquote{Lista}}) \textendash{} Sectores del grafo

\item {} 
\sphinxAtStartPar
\sphinxstyleliteralstrong{\sphinxupquote{peso}} (\sphinxstyleliteralemphasis{\sphinxupquote{string}}\sphinxstyleliteralemphasis{\sphinxupquote{, }}\sphinxstyleliteralemphasis{\sphinxupquote{optional}}) \textendash{} String con el la etiqueta peso. (Default: weight)

\end{itemize}

\end{description}\end{quote}

\end{fulllineitems}

\index{DibujarGrafoMasCorta() (en el módulo Funciones)@\spxentry{DibujarGrafoMasCorta()}\spxextra{en el módulo Funciones}}

\begin{fulllineitems}
\phantomsection\label{\detokenize{Funciones:Funciones.DibujarGrafoMasCorta}}\pysiglinewithargsret{\sphinxcode{\sphinxupquote{Funciones.}}\sphinxbfcode{\sphinxupquote{DibujarGrafoMasCorta}}}{\emph{\DUrole{n}{Grafo}}, \emph{\DUrole{n}{mostrarPesos}}, \emph{\DUrole{n}{origen}}, \emph{\DUrole{n}{destino}}, \emph{\DUrole{n}{ruta}}, \emph{\DUrole{n}{espectador}}, \emph{\DUrole{n}{sectores}}, \emph{\DUrole{n}{asiento}}, \emph{\DUrole{n}{NodoPrefinal\_pos}}, \emph{\DUrole{n}{Asiento\_pos}}, \emph{\DUrole{n}{peso}\DUrole{o}{=}\DUrole{default_value}{\textquotesingle{}weight\textquotesingle{}}}}{}
\sphinxAtStartPar
Función para dibujar el grafo con la ruta más corta
\begin{quote}\begin{description}
\item[{Parámetros}] \leavevmode\begin{itemize}
\item {} 
\sphinxAtStartPar
\sphinxstyleliteralstrong{\sphinxupquote{Grafo}} (\sphinxstyleliteralemphasis{\sphinxupquote{networkx.classes.multidigraph.MultiDiGraph}}) \textendash{} Grafo a dibujar.

\item {} 
\sphinxAtStartPar
\sphinxstyleliteralstrong{\sphinxupquote{mostrarPesos}} (\sphinxstyleliteralemphasis{\sphinxupquote{bool}}) \textendash{} Boolean \sphinxhyphen{} True si se muestran los Pesos.

\item {} 
\sphinxAtStartPar
\sphinxstyleliteralstrong{\sphinxupquote{origen}} (\sphinxstyleliteralemphasis{\sphinxupquote{string}}) \textendash{} String \sphinxhyphen{} Nodo inicial de la solucion.

\item {} 
\sphinxAtStartPar
\sphinxstyleliteralstrong{\sphinxupquote{destino}} (\sphinxstyleliteralemphasis{\sphinxupquote{string}}) \textendash{} String \sphinxhyphen{} Nodo final de la solucion.

\item {} 
\sphinxAtStartPar
\sphinxstyleliteralstrong{\sphinxupquote{ruta}} (\sphinxstyleliteralemphasis{\sphinxupquote{lista}}) \textendash{} Lista \sphinxhyphen{} (Longitud {[}Nodos\_Solucion{]})

\item {} 
\sphinxAtStartPar
\sphinxstyleliteralstrong{\sphinxupquote{espectador}} ({\hyperref[\detokenize{Clases:Clases.Espectador}]{\sphinxcrossref{\sphinxstyleliteralemphasis{\sphinxupquote{Espectador}}}}}) \textendash{} Espectador que accede a un asiento.

\item {} 
\sphinxAtStartPar
\sphinxstyleliteralstrong{\sphinxupquote{sectores}} (\sphinxstyleliteralemphasis{\sphinxupquote{Lista}}) \textendash{} Lista de Sector.

\item {} 
\sphinxAtStartPar
\sphinxstyleliteralstrong{\sphinxupquote{asiento}} ({\hyperref[\detokenize{Clases:Clases.Asiento}]{\sphinxcrossref{\sphinxstyleliteralemphasis{\sphinxupquote{Asiento}}}}}) \textendash{} Asiento del espectador.

\item {} 
\sphinxAtStartPar
\sphinxstyleliteralstrong{\sphinxupquote{NodoPrefinal\_pos}} (\sphinxstyleliteralemphasis{\sphinxupquote{List}}) \textendash{} Posición del nodo Prefinal.

\item {} 
\sphinxAtStartPar
\sphinxstyleliteralstrong{\sphinxupquote{Asiento\_pos}} (\sphinxstyleliteralemphasis{\sphinxupquote{List}}) \textendash{} Posción del asiento.

\item {} 
\sphinxAtStartPar
\sphinxstyleliteralstrong{\sphinxupquote{peso}} (\sphinxstyleliteralemphasis{\sphinxupquote{string}}\sphinxstyleliteralemphasis{\sphinxupquote{, }}\sphinxstyleliteralemphasis{\sphinxupquote{optional}}) \textendash{} String con el la etiqueta peso. (Default: weight)

\end{itemize}

\item[{Devuelve}] \leavevmode
\sphinxAtStartPar
\sphinxstylestrong{Plt donde se dibuja}

\item[{Tipo del valor devuelto}] \leavevmode
\sphinxAtStartPar
module

\end{description}\end{quote}

\end{fulllineitems}

\index{DibujarGrafoMasRapida() (en el módulo Funciones)@\spxentry{DibujarGrafoMasRapida()}\spxextra{en el módulo Funciones}}

\begin{fulllineitems}
\phantomsection\label{\detokenize{Funciones:Funciones.DibujarGrafoMasRapida}}\pysiglinewithargsret{\sphinxcode{\sphinxupquote{Funciones.}}\sphinxbfcode{\sphinxupquote{DibujarGrafoMasRapida}}}{\emph{\DUrole{n}{Grafo}}, \emph{\DUrole{n}{mostrarPesos}}, \emph{\DUrole{n}{origen}}, \emph{\DUrole{n}{destino}}, \emph{\DUrole{n}{ruta}}, \emph{\DUrole{n}{espectador}}, \emph{\DUrole{n}{sectores}}, \emph{\DUrole{n}{asiento}}, \emph{\DUrole{n}{NodoPrefinal\_pos}}, \emph{\DUrole{n}{Asiento\_pos}}, \emph{\DUrole{n}{peso}\DUrole{o}{=}\DUrole{default_value}{\textquotesingle{}weight\textquotesingle{}}}}{}
\sphinxAtStartPar
Función pará pintar el grafo con la ruta más rapida.
\begin{quote}\begin{description}
\item[{Parámetros}] \leavevmode\begin{itemize}
\item {} 
\sphinxAtStartPar
\sphinxstyleliteralstrong{\sphinxupquote{Grafo}} (\sphinxstyleliteralemphasis{\sphinxupquote{networkx.classes.multidigraph.MultiDiGraph}}) \textendash{} Grafo a dibujar.

\item {} 
\sphinxAtStartPar
\sphinxstyleliteralstrong{\sphinxupquote{mostrarPesos}} (\sphinxstyleliteralemphasis{\sphinxupquote{bool}}) \textendash{} Boolean \sphinxhyphen{} True si se muestran los Pesos.

\item {} 
\sphinxAtStartPar
\sphinxstyleliteralstrong{\sphinxupquote{origen}} (\sphinxstyleliteralemphasis{\sphinxupquote{string}}) \textendash{} String \sphinxhyphen{} Nodo inicial de la solucion.

\item {} 
\sphinxAtStartPar
\sphinxstyleliteralstrong{\sphinxupquote{destino}} (\sphinxstyleliteralemphasis{\sphinxupquote{string}}) \textendash{} String \sphinxhyphen{} Nodo final de la solucion.

\item {} 
\sphinxAtStartPar
\sphinxstyleliteralstrong{\sphinxupquote{ruta}} (\sphinxstyleliteralemphasis{\sphinxupquote{lista}}) \textendash{} Lista \sphinxhyphen{} (Longitud {[}Nodos\_Solucion{]})

\item {} 
\sphinxAtStartPar
\sphinxstyleliteralstrong{\sphinxupquote{espectador}} ({\hyperref[\detokenize{Clases:Clases.Espectador}]{\sphinxcrossref{\sphinxstyleliteralemphasis{\sphinxupquote{Espectador}}}}}) \textendash{} Espectador que accede a un asiento.

\item {} 
\sphinxAtStartPar
\sphinxstyleliteralstrong{\sphinxupquote{sectores}} (\sphinxstyleliteralemphasis{\sphinxupquote{Lista}}) \textendash{} Lista de Sector.

\item {} 
\sphinxAtStartPar
\sphinxstyleliteralstrong{\sphinxupquote{asiento}} ({\hyperref[\detokenize{Clases:Clases.Asiento}]{\sphinxcrossref{\sphinxstyleliteralemphasis{\sphinxupquote{Asiento}}}}}) \textendash{} Asiento al que se accede.

\item {} 
\sphinxAtStartPar
\sphinxstyleliteralstrong{\sphinxupquote{NodoPrefinal\_pos}} (\sphinxstyleliteralemphasis{\sphinxupquote{List}}) \textendash{} Posición del nodo Prefinal.

\item {} 
\sphinxAtStartPar
\sphinxstyleliteralstrong{\sphinxupquote{Asiento\_pos}} (\sphinxstyleliteralemphasis{\sphinxupquote{List}}) \textendash{} Posción del asiento.

\item {} 
\sphinxAtStartPar
\sphinxstyleliteralstrong{\sphinxupquote{peso}} (\sphinxstyleliteralemphasis{\sphinxupquote{string}}\sphinxstyleliteralemphasis{\sphinxupquote{, }}\sphinxstyleliteralemphasis{\sphinxupquote{optional}}) \textendash{} String con el la etiqueta peso. (Default: weight)

\end{itemize}

\item[{Devuelve}] \leavevmode
\sphinxAtStartPar
\sphinxstylestrong{Plt donde se dibuja}

\item[{Tipo del valor devuelto}] \leavevmode
\sphinxAtStartPar
module

\end{description}\end{quote}

\end{fulllineitems}

\index{DibujarGrafoAtributos() (en el módulo Funciones)@\spxentry{DibujarGrafoAtributos()}\spxextra{en el módulo Funciones}}

\begin{fulllineitems}
\phantomsection\label{\detokenize{Funciones:Funciones.DibujarGrafoAtributos}}\pysiglinewithargsret{\sphinxcode{\sphinxupquote{Funciones.}}\sphinxbfcode{\sphinxupquote{DibujarGrafoAtributos}}}{\emph{\DUrole{n}{Grafo}}, \emph{\DUrole{n}{mostrarPesos}}, \emph{\DUrole{n}{origen}}, \emph{\DUrole{n}{destino}}, \emph{\DUrole{n}{ruta}}, \emph{\DUrole{n}{espectador}}, \emph{\DUrole{n}{sectores}}, \emph{\DUrole{n}{asiento}}, \emph{\DUrole{n}{NodoPrefinal\_pos}}, \emph{\DUrole{n}{Asiento\_pos}}, \emph{\DUrole{n}{peso}\DUrole{o}{=}\DUrole{default_value}{\textquotesingle{}weight\textquotesingle{}}}}{}
\sphinxAtStartPar
Función pará pintar el grafo segun atributos.
\begin{quote}\begin{description}
\item[{Parámetros}] \leavevmode\begin{itemize}
\item {} 
\sphinxAtStartPar
\sphinxstyleliteralstrong{\sphinxupquote{Grafo}} (\sphinxstyleliteralemphasis{\sphinxupquote{networkx.classes.multidigraph.MultiDiGraph}}) \textendash{} Grafo a dibujar.

\item {} 
\sphinxAtStartPar
\sphinxstyleliteralstrong{\sphinxupquote{mostrarPesos}} (\sphinxstyleliteralemphasis{\sphinxupquote{bool}}) \textendash{} Boolean \sphinxhyphen{} True si se muestran los Pesos.

\item {} 
\sphinxAtStartPar
\sphinxstyleliteralstrong{\sphinxupquote{origen}} (\sphinxstyleliteralemphasis{\sphinxupquote{string}}) \textendash{} String \sphinxhyphen{} Nodo inicial de la solucion.

\item {} 
\sphinxAtStartPar
\sphinxstyleliteralstrong{\sphinxupquote{destino}} (\sphinxstyleliteralemphasis{\sphinxupquote{string}}) \textendash{} String \sphinxhyphen{} Nodo final de la solucion.

\item {} 
\sphinxAtStartPar
\sphinxstyleliteralstrong{\sphinxupquote{ruta}} (\sphinxstyleliteralemphasis{\sphinxupquote{lista}}) \textendash{} Lista \sphinxhyphen{} (Longitud {[}Nodos\_Solucion{]})

\item {} 
\sphinxAtStartPar
\sphinxstyleliteralstrong{\sphinxupquote{espectador}} ({\hyperref[\detokenize{Clases:Clases.Espectador}]{\sphinxcrossref{\sphinxstyleliteralemphasis{\sphinxupquote{Espectador}}}}}) \textendash{} Espectador que accede a un asiento.

\item {} 
\sphinxAtStartPar
\sphinxstyleliteralstrong{\sphinxupquote{sectores}} (\sphinxstyleliteralemphasis{\sphinxupquote{Lista}}) \textendash{} Lista de Sector.

\item {} 
\sphinxAtStartPar
\sphinxstyleliteralstrong{\sphinxupquote{asiento}} ({\hyperref[\detokenize{Clases:Clases.Asiento}]{\sphinxcrossref{\sphinxstyleliteralemphasis{\sphinxupquote{Asiento}}}}}) \textendash{} Asiento al que se accede.

\item {} 
\sphinxAtStartPar
\sphinxstyleliteralstrong{\sphinxupquote{NodoPrefinal\_pos}} (\sphinxstyleliteralemphasis{\sphinxupquote{List}}) \textendash{} Posición del nodo Prefinal.

\item {} 
\sphinxAtStartPar
\sphinxstyleliteralstrong{\sphinxupquote{Asiento\_pos}} (\sphinxstyleliteralemphasis{\sphinxupquote{List}}) \textendash{} Posción del asiento.

\item {} 
\sphinxAtStartPar
\sphinxstyleliteralstrong{\sphinxupquote{peso}} (\sphinxstyleliteralemphasis{\sphinxupquote{string}}\sphinxstyleliteralemphasis{\sphinxupquote{, }}\sphinxstyleliteralemphasis{\sphinxupquote{optional}}) \textendash{} String con el la etiqueta peso. (Default: weight)

\end{itemize}

\item[{Devuelve}] \leavevmode
\sphinxAtStartPar
\sphinxstylestrong{Plt donde se dibuja}

\item[{Tipo del valor devuelto}] \leavevmode
\sphinxAtStartPar
module

\end{description}\end{quote}

\end{fulllineitems}

\index{DibujarGrafoOcupacion() (en el módulo Funciones)@\spxentry{DibujarGrafoOcupacion()}\spxextra{en el módulo Funciones}}

\begin{fulllineitems}
\phantomsection\label{\detokenize{Funciones:Funciones.DibujarGrafoOcupacion}}\pysiglinewithargsret{\sphinxcode{\sphinxupquote{Funciones.}}\sphinxbfcode{\sphinxupquote{DibujarGrafoOcupacion}}}{\emph{\DUrole{n}{Grafo}}, \emph{\DUrole{n}{mostrarPesos}}, \emph{\DUrole{n}{origen}}, \emph{\DUrole{n}{destino}}, \emph{\DUrole{n}{espectador}}, \emph{\DUrole{n}{sectores}}, \emph{\DUrole{n}{asiento}}, \emph{\DUrole{n}{nodoParado}}, \emph{\DUrole{n}{dibujarEnlacePrefinalAsiento}\DUrole{o}{=}\DUrole{default_value}{False}}, \emph{\DUrole{n}{NodoPrefinal\_pos}\DUrole{o}{=}\DUrole{default_value}{None}}, \emph{\DUrole{n}{Asiento\_pos}\DUrole{o}{=}\DUrole{default_value}{None}}, \emph{\DUrole{n}{peso}\DUrole{o}{=}\DUrole{default_value}{\textquotesingle{}weight\textquotesingle{}}}}{}
\sphinxAtStartPar
Funcion para dibujar el grafo de atributos
\begin{quote}\begin{description}
\item[{Parámetros}] \leavevmode\begin{itemize}
\item {} 
\sphinxAtStartPar
\sphinxstyleliteralstrong{\sphinxupquote{Grafo}} (\sphinxstyleliteralemphasis{\sphinxupquote{networkx.classes.multidigraph.MultiDiGraph}}) \textendash{} Grafo a dibujar.

\item {} 
\sphinxAtStartPar
\sphinxstyleliteralstrong{\sphinxupquote{mostrarPesos}} (\sphinxstyleliteralemphasis{\sphinxupquote{bool}}) \textendash{} Boolean \sphinxhyphen{} True si se muestran los Pesos.

\item {} 
\sphinxAtStartPar
\sphinxstyleliteralstrong{\sphinxupquote{origen}} (\sphinxstyleliteralemphasis{\sphinxupquote{string}}) \textendash{} String \sphinxhyphen{} Nodo inicial de la solucion.

\item {} 
\sphinxAtStartPar
\sphinxstyleliteralstrong{\sphinxupquote{destino}} (\sphinxstyleliteralemphasis{\sphinxupquote{string}}) \textendash{} String \sphinxhyphen{} Nodo final de la solucion.

\item {} 
\sphinxAtStartPar
\sphinxstyleliteralstrong{\sphinxupquote{espectador}} ({\hyperref[\detokenize{Clases:Clases.Espectador}]{\sphinxcrossref{\sphinxstyleliteralemphasis{\sphinxupquote{Espectador}}}}}) \textendash{} Espectador que accede a un asiento.

\item {} 
\sphinxAtStartPar
\sphinxstyleliteralstrong{\sphinxupquote{sectores}} (\sphinxstyleliteralemphasis{\sphinxupquote{Lista}}) \textendash{} Lista de Sector.

\item {} 
\sphinxAtStartPar
\sphinxstyleliteralstrong{\sphinxupquote{asiento}} ({\hyperref[\detokenize{Clases:Clases.Asiento}]{\sphinxcrossref{\sphinxstyleliteralemphasis{\sphinxupquote{Asiento}}}}}) \textendash{} Asiento al que se accede.

\item {} 
\sphinxAtStartPar
\sphinxstyleliteralstrong{\sphinxupquote{nodoParado}} (\sphinxstyleliteralemphasis{\sphinxupquote{String}}) \textendash{} Nodo en el que esta parado el usuario.

\item {} 
\sphinxAtStartPar
\sphinxstyleliteralstrong{\sphinxupquote{dibujarEnlacePrefinalAsiento}} (\sphinxstyleliteralemphasis{\sphinxupquote{Boolean}}\sphinxstyleliteralemphasis{\sphinxupquote{, }}\sphinxstyleliteralemphasis{\sphinxupquote{optional}}) \textendash{} Bool para dibujar el enlace del nodo prefinal al asiento. (Default: False)

\item {} 
\sphinxAtStartPar
\sphinxstyleliteralstrong{\sphinxupquote{NodoPrefinal\_pos}} (\sphinxstyleliteralemphasis{\sphinxupquote{Tupla}}\sphinxstyleliteralemphasis{\sphinxupquote{, }}\sphinxstyleliteralemphasis{\sphinxupquote{optional}}) \textendash{} Posicion del nodo Prefinal. (Default: None)

\item {} 
\sphinxAtStartPar
\sphinxstyleliteralstrong{\sphinxupquote{Asiento\_pos}} (\sphinxstyleliteralemphasis{\sphinxupquote{Tupla}}\sphinxstyleliteralemphasis{\sphinxupquote{, }}\sphinxstyleliteralemphasis{\sphinxupquote{optional}}) \textendash{} Posicion del asiento. (Default: None)

\item {} 
\sphinxAtStartPar
\sphinxstyleliteralstrong{\sphinxupquote{peso}} (\sphinxstyleliteralemphasis{\sphinxupquote{string}}\sphinxstyleliteralemphasis{\sphinxupquote{, }}\sphinxstyleliteralemphasis{\sphinxupquote{optional}}) \textendash{} String con el la etiqueta peso. (Default: weight)

\end{itemize}

\item[{Devuelve}] \leavevmode
\sphinxAtStartPar


\item[{Tipo del valor devuelto}] \leavevmode
\sphinxAtStartPar
Plt donde se dibujo.

\end{description}\end{quote}

\end{fulllineitems}

\index{dibujarAsientoGrafoGeneral() (en el módulo Funciones)@\spxentry{dibujarAsientoGrafoGeneral()}\spxextra{en el módulo Funciones}}

\begin{fulllineitems}
\phantomsection\label{\detokenize{Funciones:Funciones.dibujarAsientoGrafoGeneral}}\pysiglinewithargsret{\sphinxcode{\sphinxupquote{Funciones.}}\sphinxbfcode{\sphinxupquote{dibujarAsientoGrafoGeneral}}}{\emph{\DUrole{n}{posicionX\_asiento}}, \emph{\DUrole{n}{posicionY\_asiento}}, \emph{\DUrole{n}{plt}}}{}
\end{fulllineitems}

\index{addnodoPrefinal() (en el módulo Funciones)@\spxentry{addnodoPrefinal()}\spxextra{en el módulo Funciones}}

\begin{fulllineitems}
\phantomsection\label{\detokenize{Funciones:Funciones.addnodoPrefinal}}\pysiglinewithargsret{\sphinxcode{\sphinxupquote{Funciones.}}\sphinxbfcode{\sphinxupquote{addnodoPrefinal}}}{\emph{\DUrole{n}{G}}, \emph{\DUrole{n}{posicionUnion}}, \emph{\DUrole{n}{sectorEspectador}}, \emph{\DUrole{n}{enlacePreFinal}}, \emph{\DUrole{n}{fila}}, \emph{\DUrole{n}{nombreNodoPrefinal}\DUrole{o}{=}\DUrole{default_value}{\textquotesingle{}NP\textquotesingle{}}}}{}
\sphinxAtStartPar
Función para agregar al Grafo un nodo prefinal equivalente a la proyección del asiento en las escaleras.
\begin{quote}\begin{description}
\item[{Parámetros}] \leavevmode\begin{itemize}
\item {} 
\sphinxAtStartPar
\sphinxstyleliteralstrong{\sphinxupquote{G}} (\sphinxstyleliteralemphasis{\sphinxupquote{networkx.classes.multidigraph.MultiDiGraph}}) \textendash{} Grafo a actualizar.

\item {} 
\sphinxAtStartPar
\sphinxstyleliteralstrong{\sphinxupquote{posicionUnion}} (\sphinxstyleliteralemphasis{\sphinxupquote{Tupla}}) \textendash{} Posicion de la proyeccion del asiento sobre el enlace de su derecha o izquierda.

\item {} 
\sphinxAtStartPar
\sphinxstyleliteralstrong{\sphinxupquote{sectorEspectador}} ({\hyperref[\detokenize{Clases:Clases.Sector}]{\sphinxcrossref{\sphinxstyleliteralemphasis{\sphinxupquote{Sector}}}}}) \textendash{} Sector sobre el que se encuentra el asiento del espectador.

\item {} 
\sphinxAtStartPar
\sphinxstyleliteralstrong{\sphinxupquote{enlacePreFinal}} (\sphinxstyleliteralemphasis{\sphinxupquote{Tupla}}) \textendash{} Enlace actual del grafo que se borrara para agregar el nuevo nodo Prefinal.

\item {} 
\sphinxAtStartPar
\sphinxstyleliteralstrong{\sphinxupquote{fila}} (\sphinxstyleliteralemphasis{\sphinxupquote{Integer}}) \textendash{} Valor de la fila donde esta el asiento.

\item {} 
\sphinxAtStartPar
\sphinxstyleliteralstrong{\sphinxupquote{nombreNodoPrefinal}} (\sphinxstyleliteralemphasis{\sphinxupquote{String}}\sphinxstyleliteralemphasis{\sphinxupquote{, }}\sphinxstyleliteralemphasis{\sphinxupquote{optional}}) \textendash{} Nombre del nodo prefinal resultante. Valor por defecto “NP”.

\end{itemize}

\item[{Devuelve}] \leavevmode
\sphinxAtStartPar
\sphinxstylestrong{nombreNodoPrefinal} \textendash{} Nombre del nodo prefinal.

\item[{Tipo del valor devuelto}] \leavevmode
\sphinxAtStartPar
String

\end{description}\end{quote}

\end{fulllineitems}

\index{dibujarRectaNodoPrefinal\_Asiento() (en el módulo Funciones)@\spxentry{dibujarRectaNodoPrefinal\_Asiento()}\spxextra{en el módulo Funciones}}

\begin{fulllineitems}
\phantomsection\label{\detokenize{Funciones:Funciones.dibujarRectaNodoPrefinal_Asiento}}\pysiglinewithargsret{\sphinxcode{\sphinxupquote{Funciones.}}\sphinxbfcode{\sphinxupquote{dibujarRectaNodoPrefinal\_Asiento}}}{\emph{\DUrole{n}{NodoPrefinal\_pos}}, \emph{\DUrole{n}{Asiento\_pos}}, \emph{\DUrole{n}{plot}}}{}
\sphinxAtStartPar
Funcion para dibujar una línea que emule el camino desde la escalera hasta el asiento.
\begin{quote}\begin{description}
\item[{Parámetros}] \leavevmode\begin{itemize}
\item {} 
\sphinxAtStartPar
\sphinxstyleliteralstrong{\sphinxupquote{NodoPrefinal\_pos}} (\sphinxstyleliteralemphasis{\sphinxupquote{Tupla}}) \textendash{} Posiciones del nodo Prefinal.

\item {} 
\sphinxAtStartPar
\sphinxstyleliteralstrong{\sphinxupquote{Asiento\_pos}} (\sphinxstyleliteralemphasis{\sphinxupquote{Tupla}}) \textendash{} Posiciones del asiento.

\item {} 
\sphinxAtStartPar
\sphinxstyleliteralstrong{\sphinxupquote{plot}} (\sphinxstyleliteralemphasis{\sphinxupquote{module}}) \textendash{} Plt donde se dubujara la linea.

\end{itemize}

\end{description}\end{quote}

\end{fulllineitems}

\index{dibujarDatosEspectadorGeneral() (en el módulo Funciones)@\spxentry{dibujarDatosEspectadorGeneral()}\spxextra{en el módulo Funciones}}

\begin{fulllineitems}
\phantomsection\label{\detokenize{Funciones:Funciones.dibujarDatosEspectadorGeneral}}\pysiglinewithargsret{\sphinxcode{\sphinxupquote{Funciones.}}\sphinxbfcode{\sphinxupquote{dibujarDatosEspectadorGeneral}}}{\emph{\DUrole{n}{Grafo}}, \emph{\DUrole{n}{espectador}}, \emph{\DUrole{n}{plot}}}{}
\sphinxAtStartPar
Función para dibujar los datos del espectador en el plot
\begin{quote}\begin{description}
\item[{Parámetros}] \leavevmode\begin{itemize}
\item {} 
\sphinxAtStartPar
\sphinxstyleliteralstrong{\sphinxupquote{Grafo}} (\sphinxstyleliteralemphasis{\sphinxupquote{networkx.classes.multidigraph.MultiDiGraph}}) \textendash{} Grafo General.

\item {} 
\sphinxAtStartPar
\sphinxstyleliteralstrong{\sphinxupquote{espectador}} ({\hyperref[\detokenize{Clases:Clases.Espectador}]{\sphinxcrossref{\sphinxstyleliteralemphasis{\sphinxupquote{Espectador}}}}}) \textendash{} Espectador.

\item {} 
\sphinxAtStartPar
\sphinxstyleliteralstrong{\sphinxupquote{plot}} (\sphinxstyleliteralemphasis{\sphinxupquote{mdoule}}) \textendash{} Plt donde se dibuja.

\end{itemize}

\end{description}\end{quote}

\end{fulllineitems}

\index{dibujarDatosEspectadorMasCorta() (en el módulo Funciones)@\spxentry{dibujarDatosEspectadorMasCorta()}\spxextra{en el módulo Funciones}}

\begin{fulllineitems}
\phantomsection\label{\detokenize{Funciones:Funciones.dibujarDatosEspectadorMasCorta}}\pysiglinewithargsret{\sphinxcode{\sphinxupquote{Funciones.}}\sphinxbfcode{\sphinxupquote{dibujarDatosEspectadorMasCorta}}}{\emph{\DUrole{n}{Grafo}}, \emph{\DUrole{n}{espectador}}, \emph{\DUrole{n}{plot}}}{}
\sphinxAtStartPar
Función para dibujar los datos del espectador en el plot
\begin{quote}\begin{description}
\item[{Parámetros}] \leavevmode\begin{itemize}
\item {} 
\sphinxAtStartPar
\sphinxstyleliteralstrong{\sphinxupquote{Grafo}} (\sphinxstyleliteralemphasis{\sphinxupquote{networkx.classes.multidigraph.MultiDiGraph}}) \textendash{} Grafo General.

\item {} 
\sphinxAtStartPar
\sphinxstyleliteralstrong{\sphinxupquote{espectador}} ({\hyperref[\detokenize{Clases:Clases.Espectador}]{\sphinxcrossref{\sphinxstyleliteralemphasis{\sphinxupquote{Espectador}}}}}) \textendash{} Espectador.

\item {} 
\sphinxAtStartPar
\sphinxstyleliteralstrong{\sphinxupquote{plot}} (\sphinxstyleliteralemphasis{\sphinxupquote{mdoule}}) \textendash{} Plt donde se dibuja.

\end{itemize}

\end{description}\end{quote}

\end{fulllineitems}

\index{dibujarDatosEspectadorMasRapida() (en el módulo Funciones)@\spxentry{dibujarDatosEspectadorMasRapida()}\spxextra{en el módulo Funciones}}

\begin{fulllineitems}
\phantomsection\label{\detokenize{Funciones:Funciones.dibujarDatosEspectadorMasRapida}}\pysiglinewithargsret{\sphinxcode{\sphinxupquote{Funciones.}}\sphinxbfcode{\sphinxupquote{dibujarDatosEspectadorMasRapida}}}{\emph{\DUrole{n}{Grafo}}, \emph{\DUrole{n}{espectador}}, \emph{\DUrole{n}{plot}}}{}
\sphinxAtStartPar
Función para dibujar los datos del espectador en el plot
\begin{quote}\begin{description}
\item[{Parámetros}] \leavevmode\begin{itemize}
\item {} 
\sphinxAtStartPar
\sphinxstyleliteralstrong{\sphinxupquote{Grafo}} (\sphinxstyleliteralemphasis{\sphinxupquote{networkx.classes.multidigraph.MultiDiGraph}}) \textendash{} Grafo General.

\item {} 
\sphinxAtStartPar
\sphinxstyleliteralstrong{\sphinxupquote{espectador}} ({\hyperref[\detokenize{Clases:Clases.Espectador}]{\sphinxcrossref{\sphinxstyleliteralemphasis{\sphinxupquote{Espectador}}}}}) \textendash{} Espectador.

\item {} 
\sphinxAtStartPar
\sphinxstyleliteralstrong{\sphinxupquote{plot}} (\sphinxstyleliteralemphasis{\sphinxupquote{mdoule}}) \textendash{} Plt donde se dibuja.

\end{itemize}

\end{description}\end{quote}

\end{fulllineitems}

\index{dibujarDatosEspectadorAtributos() (en el módulo Funciones)@\spxentry{dibujarDatosEspectadorAtributos()}\spxextra{en el módulo Funciones}}

\begin{fulllineitems}
\phantomsection\label{\detokenize{Funciones:Funciones.dibujarDatosEspectadorAtributos}}\pysiglinewithargsret{\sphinxcode{\sphinxupquote{Funciones.}}\sphinxbfcode{\sphinxupquote{dibujarDatosEspectadorAtributos}}}{\emph{\DUrole{n}{Grafo}}, \emph{\DUrole{n}{espectador}}, \emph{\DUrole{n}{plot}}}{}
\sphinxAtStartPar
Función para dibujar los datos del espectador en el plot
\begin{quote}\begin{description}
\item[{Parámetros}] \leavevmode\begin{itemize}
\item {} 
\sphinxAtStartPar
\sphinxstyleliteralstrong{\sphinxupquote{Grafo}} (\sphinxstyleliteralemphasis{\sphinxupquote{networkx.classes.multidigraph.MultiDiGraph}}) \textendash{} Grafo General.

\item {} 
\sphinxAtStartPar
\sphinxstyleliteralstrong{\sphinxupquote{espectador}} ({\hyperref[\detokenize{Clases:Clases.Espectador}]{\sphinxcrossref{\sphinxstyleliteralemphasis{\sphinxupquote{Espectador}}}}}) \textendash{} Espectador.

\item {} 
\sphinxAtStartPar
\sphinxstyleliteralstrong{\sphinxupquote{plot}} (\sphinxstyleliteralemphasis{\sphinxupquote{mdoule}}) \textendash{} Plt donde se dibuja.

\end{itemize}

\end{description}\end{quote}

\end{fulllineitems}

\index{botonAtributosPersonalesClick() (en el módulo Funciones)@\spxentry{botonAtributosPersonalesClick()}\spxextra{en el módulo Funciones}}

\begin{fulllineitems}
\phantomsection\label{\detokenize{Funciones:Funciones.botonAtributosPersonalesClick}}\pysiglinewithargsret{\sphinxcode{\sphinxupquote{Funciones.}}\sphinxbfcode{\sphinxupquote{botonAtributosPersonalesClick}}}{\emph{\DUrole{n}{ventana\_self}}, \emph{\DUrole{n}{escalerasConBarandillas}}, \emph{\DUrole{n}{escalerasEnBuenEstado}}, \emph{\DUrole{n}{pasillosAmplios}}, \emph{\DUrole{n}{pasillosVentilados}}, \emph{\DUrole{n}{pasillosSecos}}, \emph{\DUrole{n}{pasilloIluminados}}}{}
\sphinxAtStartPar
Metodo auxiliar para capturar el click del boton Confirmar en el formulario de atributos

\sphinxAtStartPar
y modificar el diccionario respuestas

\end{fulllineitems}

\index{leerSectoresCSV() (en el módulo Funciones)@\spxentry{leerSectoresCSV()}\spxextra{en el módulo Funciones}}

\begin{fulllineitems}
\phantomsection\label{\detokenize{Funciones:Funciones.leerSectoresCSV}}\pysiglinewithargsret{\sphinxcode{\sphinxupquote{Funciones.}}\sphinxbfcode{\sphinxupquote{leerSectoresCSV}}}{\emph{\DUrole{n}{fichero}}, \emph{\DUrole{n}{Grafo}}}{}
\sphinxAtStartPar
Función para leer los Sectores de la grada de un fichero CSV
\begin{quote}\begin{description}
\item[{Parámetros}] \leavevmode\begin{itemize}
\item {} 
\sphinxAtStartPar
\sphinxstyleliteralstrong{\sphinxupquote{fichero}} (\sphinxstyleliteralemphasis{\sphinxupquote{String}}) \textendash{} Fichero .csv con los sectores.

\item {} 
\sphinxAtStartPar
\sphinxstyleliteralstrong{\sphinxupquote{Grafo}} (\sphinxstyleliteralemphasis{\sphinxupquote{networkx.classes.multidigraph.MultiDiGraph}}) \textendash{} Grafo del estadio.

\end{itemize}

\item[{Devuelve}] \leavevmode
\sphinxAtStartPar
\sphinxstylestrong{sectores} \textendash{} Lista con los sectores leidos.

\item[{Tipo del valor devuelto}] \leavevmode
\sphinxAtStartPar
Lista

\end{description}\end{quote}

\end{fulllineitems}

\index{importarGrafoJSON() (en el módulo Funciones)@\spxentry{importarGrafoJSON()}\spxextra{en el módulo Funciones}}

\begin{fulllineitems}
\phantomsection\label{\detokenize{Funciones:Funciones.importarGrafoJSON}}\pysiglinewithargsret{\sphinxcode{\sphinxupquote{Funciones.}}\sphinxbfcode{\sphinxupquote{importarGrafoJSON}}}{\emph{\DUrole{n}{Fichero}}}{}
\sphinxAtStartPar
Funcion para leer el grafo a parti de un fichero JSON
\begin{quote}\begin{description}
\item[{Parámetros}] \leavevmode
\sphinxAtStartPar
\sphinxstyleliteralstrong{\sphinxupquote{Fichero}} (\sphinxstyleliteralemphasis{\sphinxupquote{String}}) \textendash{} Fichero de extension .json.

\item[{Devuelve}] \leavevmode
\sphinxAtStartPar
\sphinxstylestrong{Grafo} \textendash{} Grafo leido del fichero JSON.

\item[{Tipo del valor devuelto}] \leavevmode
\sphinxAtStartPar
networkx.classes.multidigraph.MultiDiGraph

\end{description}\end{quote}

\end{fulllineitems}

\index{exportarGrafoJSON() (en el módulo Funciones)@\spxentry{exportarGrafoJSON()}\spxextra{en el módulo Funciones}}

\begin{fulllineitems}
\phantomsection\label{\detokenize{Funciones:Funciones.exportarGrafoJSON}}\pysiglinewithargsret{\sphinxcode{\sphinxupquote{Funciones.}}\sphinxbfcode{\sphinxupquote{exportarGrafoJSON}}}{\emph{\DUrole{n}{Grafo}}}{}
\sphinxAtStartPar
Funcion para generar un fichero JSON a partir del grafo dado.
\begin{quote}\begin{description}
\item[{Parámetros}] \leavevmode
\sphinxAtStartPar
\sphinxstyleliteralstrong{\sphinxupquote{Grafo}} (\sphinxstyleliteralemphasis{\sphinxupquote{networkx.classes.multidigraph.MultiDiGraph}}) \textendash{} Grafo a exportar.

\item[{Devuelve}] \leavevmode
\sphinxAtStartPar


\item[{Tipo del valor devuelto}] \leavevmode
\sphinxAtStartPar
None.

\end{description}\end{quote}

\end{fulllineitems}

\index{MenuAtributosPersonales (clase en Funciones)@\spxentry{MenuAtributosPersonales}\spxextra{clase en Funciones}}

\begin{fulllineitems}
\phantomsection\label{\detokenize{Funciones:Funciones.MenuAtributosPersonales}}\pysiglinewithargsret{\sphinxbfcode{\sphinxupquote{class }}\sphinxcode{\sphinxupquote{Funciones.}}\sphinxbfcode{\sphinxupquote{MenuAtributosPersonales}}}{\emph{\DUrole{n}{master}}, \emph{\DUrole{n}{status}}, \emph{\DUrole{o}{*}\DUrole{n}{options}}}{}
\sphinxAtStartPar
Clase para crear un menú de atributos a puntuar

\end{fulllineitems}

\index{Respuestas (clase en Funciones)@\spxentry{Respuestas}\spxextra{clase en Funciones}}

\begin{fulllineitems}
\phantomsection\label{\detokenize{Funciones:Funciones.Respuestas}}\pysigline{\sphinxbfcode{\sphinxupquote{class }}\sphinxcode{\sphinxupquote{Funciones.}}\sphinxbfcode{\sphinxupquote{Respuestas}}}
\sphinxAtStartPar
Clase para las respuestas del menú de atributos
\index{rellenar() (método de Funciones.Respuestas)@\spxentry{rellenar()}\spxextra{método de Funciones.Respuestas}}

\begin{fulllineitems}
\phantomsection\label{\detokenize{Funciones:Funciones.Respuestas.rellenar}}\pysiglinewithargsret{\sphinxbfcode{\sphinxupquote{rellenar}}}{\emph{\DUrole{n}{escalerasConBarandillas}}, \emph{\DUrole{n}{escalerasEnBuenEstado}}, \emph{\DUrole{n}{pasillosAmplios}}, \emph{\DUrole{n}{pasillosVentilados}}, \emph{\DUrole{n}{pasillosSecos}}, \emph{\DUrole{n}{pasilloIluminados}}}{}
\sphinxAtStartPar
Función para rellenar los campos de las respuestas
\begin{quote}\begin{description}
\item[{Parámetros}] \leavevmode\begin{itemize}
\item {} 
\sphinxAtStartPar
\sphinxstyleliteralstrong{\sphinxupquote{escalerasConBarandillas}} (\sphinxstyleliteralemphasis{\sphinxupquote{Integer}}) \textendash{} Valor respondido para este campo.

\item {} 
\sphinxAtStartPar
\sphinxstyleliteralstrong{\sphinxupquote{escalerasEnBuenEstado}} (\sphinxstyleliteralemphasis{\sphinxupquote{Integer}}) \textendash{} Valor respondido para este campo.

\item {} 
\sphinxAtStartPar
\sphinxstyleliteralstrong{\sphinxupquote{pasillosAmplios}} (\sphinxstyleliteralemphasis{\sphinxupquote{Integer}}) \textendash{} Valor respondido para este campo.

\item {} 
\sphinxAtStartPar
\sphinxstyleliteralstrong{\sphinxupquote{pasillosVentilados}} (\sphinxstyleliteralemphasis{\sphinxupquote{Integer}}) \textendash{} Valor respondido para este campo.

\item {} 
\sphinxAtStartPar
\sphinxstyleliteralstrong{\sphinxupquote{pasillosSecos}} (\sphinxstyleliteralemphasis{\sphinxupquote{Integer}}) \textendash{} Valor respondido para este campo.

\item {} 
\sphinxAtStartPar
\sphinxstyleliteralstrong{\sphinxupquote{pasilloIluminados}} (\sphinxstyleliteralemphasis{\sphinxupquote{Integer}}) \textendash{} Valor respondido para este campo.

\end{itemize}

\item[{Devuelve}] \leavevmode
\sphinxAtStartPar


\item[{Tipo del valor devuelto}] \leavevmode
\sphinxAtStartPar
None.

\end{description}\end{quote}

\end{fulllineitems}


\end{fulllineitems}

\index{ventanaAtributosPersonales (clase en Funciones)@\spxentry{ventanaAtributosPersonales}\spxextra{clase en Funciones}}

\begin{fulllineitems}
\phantomsection\label{\detokenize{Funciones:Funciones.ventanaAtributosPersonales}}\pysigline{\sphinxbfcode{\sphinxupquote{class }}\sphinxcode{\sphinxupquote{Funciones.}}\sphinxbfcode{\sphinxupquote{ventanaAtributosPersonales}}}
\sphinxAtStartPar
Clase para dibujar una ventana con los atributos a puntuar
\index{center() (método de Funciones.ventanaAtributosPersonales)@\spxentry{center()}\spxextra{método de Funciones.ventanaAtributosPersonales}}

\begin{fulllineitems}
\phantomsection\label{\detokenize{Funciones:Funciones.ventanaAtributosPersonales.center}}\pysiglinewithargsret{\sphinxbfcode{\sphinxupquote{center}}}{\emph{\DUrole{n}{window}}}{}
\sphinxAtStartPar
Función para centrar la ventana

\end{fulllineitems}


\end{fulllineitems}



\chapter{Clases}
\label{\detokenize{Clases:clases}}\label{\detokenize{Clases::doc}}
\sphinxAtStartPar
Módulo con las clases usadas en el proyecto.

\phantomsection\label{\detokenize{Clases:module-Clases}}\index{módulo@\spxentry{módulo}!Clases@\spxentry{Clases}}\index{Clases@\spxentry{Clases}!módulo@\spxentry{módulo}}
\sphinxAtStartPar
@author: Pelayo Tiesta
\index{Sector (clase en Clases)@\spxentry{Sector}\spxextra{clase en Clases}}

\begin{fulllineitems}
\phantomsection\label{\detokenize{Clases:Clases.Sector}}\pysiglinewithargsret{\sphinxbfcode{\sphinxupquote{class }}\sphinxcode{\sphinxupquote{Clases.}}\sphinxbfcode{\sphinxupquote{Sector}}}{\emph{\DUrole{n}{nombre}}, \emph{\DUrole{n}{nodoArribaIzquierda}}, \emph{\DUrole{n}{nodoArribaDerecha}}, \emph{\DUrole{n}{nodoAbajoIzquierda}}, \emph{\DUrole{n}{nodoAbajoDerecha}}, \emph{\DUrole{n}{filas}}, \emph{\DUrole{n}{columnas}}, \emph{\DUrole{n}{posicion}}}{}
\sphinxAtStartPar
Clase para crear un objeto sector en el estadio

\end{fulllineitems}

\index{Asiento (clase en Clases)@\spxentry{Asiento}\spxextra{clase en Clases}}

\begin{fulllineitems}
\phantomsection\label{\detokenize{Clases:Clases.Asiento}}\pysiglinewithargsret{\sphinxbfcode{\sphinxupquote{class }}\sphinxcode{\sphinxupquote{Clases.}}\sphinxbfcode{\sphinxupquote{Asiento}}}{\emph{\DUrole{n}{sector}}, \emph{\DUrole{n}{fila}}, \emph{\DUrole{n}{columna}}}{}
\sphinxAtStartPar
Clase para crear un objeto asiento ubicado en el estadio
\index{enlaceProyeccion() (método de Clases.Asiento)@\spxentry{enlaceProyeccion()}\spxextra{método de Clases.Asiento}}

\begin{fulllineitems}
\phantomsection\label{\detokenize{Clases:Clases.Asiento.enlaceProyeccion}}\pysiglinewithargsret{\sphinxbfcode{\sphinxupquote{enlaceProyeccion}}}{}{}
\sphinxAtStartPar
Función para obtener el enlace de la proyección del asiento.
\begin{quote}\begin{description}
\item[{Muestra}] \leavevmode
\sphinxAtStartPar
\sphinxstyleliteralstrong{\sphinxupquote{Exception}} \textendash{} Produce Excepcion si no existe pasillo derecho ni izquierdo.

\item[{Devuelve}] \leavevmode
\sphinxAtStartPar


\item[{Tipo del valor devuelto}] \leavevmode
\sphinxAtStartPar
Enlace \sphinxhyphen{} Tupla

\end{description}\end{quote}

\end{fulllineitems}

\index{nodoPrePreFinal() (método de Clases.Asiento)@\spxentry{nodoPrePreFinal()}\spxextra{método de Clases.Asiento}}

\begin{fulllineitems}
\phantomsection\label{\detokenize{Clases:Clases.Asiento.nodoPrePreFinal}}\pysiglinewithargsret{\sphinxbfcode{\sphinxupquote{nodoPrePreFinal}}}{}{}
\sphinxAtStartPar
Función para rellenar el nodoPrePrefinal
Se calcula primero si se ira al lado derecho o izquierdo en funcion de la ubicacion horizontal del asiento.
Posteriormente se calcula si el nodoPrePrefinal sera el nodo superior o inferior
\begin{quote}\begin{description}
\item[{Muestra}] \leavevmode
\sphinxAtStartPar
\sphinxstyleliteralstrong{\sphinxupquote{Exception}} \textendash{} Se produce si no tiene enlace derecho o izquierdo (Asiento no accesible).

\item[{Devuelve}] \leavevmode
\sphinxAtStartPar


\item[{Tipo del valor devuelto}] \leavevmode
\sphinxAtStartPar
String \sphinxhyphen{} Nodo PrePrefinal

\end{description}\end{quote}

\end{fulllineitems}


\end{fulllineitems}

\index{Espectador (clase en Clases)@\spxentry{Espectador}\spxextra{clase en Clases}}

\begin{fulllineitems}
\phantomsection\label{\detokenize{Clases:Clases.Espectador}}\pysiglinewithargsret{\sphinxbfcode{\sphinxupquote{class }}\sphinxcode{\sphinxupquote{Clases.}}\sphinxbfcode{\sphinxupquote{Espectador}}}{\emph{\DUrole{n}{ID}}, \emph{\DUrole{n}{asiento}}, \emph{\DUrole{n}{puertaEntrada}}}{}
\sphinxAtStartPar
Clase para crear un objeto espectador al que se recomendara una ruta

\end{fulllineitems}

\index{ThreadingOcupacion (clase en Clases)@\spxentry{ThreadingOcupacion}\spxextra{clase en Clases}}

\begin{fulllineitems}
\phantomsection\label{\detokenize{Clases:Clases.ThreadingOcupacion}}\pysiglinewithargsret{\sphinxbfcode{\sphinxupquote{class }}\sphinxcode{\sphinxupquote{Clases.}}\sphinxbfcode{\sphinxupquote{ThreadingOcupacion}}}{\emph{\DUrole{n}{Grafo}}}{}
\sphinxAtStartPar
Clase para crear un hilo que simule la entrada de personas en la grada
\index{run() (método de Clases.ThreadingOcupacion)@\spxentry{run()}\spxextra{método de Clases.ThreadingOcupacion}}

\begin{fulllineitems}
\phantomsection\label{\detokenize{Clases:Clases.ThreadingOcupacion.run}}\pysiglinewithargsret{\sphinxbfcode{\sphinxupquote{run}}}{}{}
\sphinxAtStartPar
Metodo para corren en segundo plano y modificar el peso del grafo con la ocupacion

\end{fulllineitems}

\index{terminate() (método de Clases.ThreadingOcupacion)@\spxentry{terminate()}\spxextra{método de Clases.ThreadingOcupacion}}

\begin{fulllineitems}
\phantomsection\label{\detokenize{Clases:Clases.ThreadingOcupacion.terminate}}\pysiglinewithargsret{\sphinxbfcode{\sphinxupquote{terminate}}}{}{}
\sphinxAtStartPar
Función para terminar el hilo

\end{fulllineitems}


\end{fulllineitems}



\chapter{Información Adicional}
\label{\detokenize{Informacion_Adicional:informacion-adicional}}\label{\detokenize{Informacion_Adicional::doc}}

\section{Datos}
\label{\detokenize{Informacion_Adicional:datos}}

\subsection{Título del proyecto}
\label{\detokenize{Informacion_Adicional:titulo-del-proyecto}}
\sphinxAtStartPar
SISTEMA DE RECOMENDACIONES PARA ACCESOS A ESTADIOS DE FÚTBOL


\subsection{Autor}
\label{\detokenize{Informacion_Adicional:autor}}
\sphinxAtStartPar
\sphinxhref{https://www.linkedin.com/in/pelayo-tiesta/}{Pelayo Ricardo Tiesta Cosío}


\subsection{Centro académico}
\label{\detokenize{Informacion_Adicional:centro-academico}}
\sphinxAtStartPar
\sphinxhref{https://www.uniovi.es}{Universidad de Oviedo}

\sphinxAtStartPar
\sphinxhref{https://epigijon.uniovi.es}{Escuela Politécnica de Ingeniería de Gijón}


\chapter{Código Fuente}
\label{\detokenize{index:codigo-fuente}}
\sphinxAtStartPar
El código fuente de este proyecto se encuentra en el repositorio de \sphinxhref{https://github.com/Pelayin97/CODIGO-TFG-RUTAS}{GitHub}.


\renewcommand{\indexname}{Índice de Módulos Python}
\begin{sphinxtheindex}
\let\bigletter\sphinxstyleindexlettergroup
\bigletter{c}
\item\relax\sphinxstyleindexentry{Clases}\sphinxstyleindexpageref{Clases:\detokenize{module-Clases}}
\indexspace
\bigletter{f}
\item\relax\sphinxstyleindexentry{Funciones}\sphinxstyleindexpageref{Funciones:\detokenize{module-Funciones}}
\end{sphinxtheindex}

\renewcommand{\indexname}{Índice}
\printindex
\end{document}